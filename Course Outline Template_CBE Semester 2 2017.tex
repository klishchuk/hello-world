\documentclass[11pt]{article}

\usepackage{sectsty}
\usepackage{helvet}
\usepackage{hyperref}
\usepackage{enumerate}
\usepackage{titling}
\usepackage[dvipsnames]{xcolor}
\usepackage{graphicx}
\usepackage{ulem}
\usepackage[margin=1in]{geometry}
 
\pagestyle{empty}

\hypersetup{colorlinks=false, allcolors=blue, linkcolor=blue, urlbordercolor=blue, pdfborderstyle={/S/U/W 1}}
\renewcommand{\familydefault}{\sfdefault}
\renewcommand\UrlFont{\color{blue}\sffamily\underline}

\sectionfont{\large\bf}
\subsectionfont{\normalfont\bf\large\underline}

%%% TITLE %%%
\title{\bfseries \LARGE  [Course Code] \\
	\Huge [Title of Course] \hrule height 3pt \vspace{-2cm}}

\date{}
\author{}


\begin{document}
\setlength{\parindent}{0cm}

\textbf{{\color{red}\small Note : Red text is deleted as sections are completed. The font has been selected to help make the guide accessible to students with disabilities who use assistive technologies. \vspace{-1.2cm}}} \\

\begin{minipage}{\textwidth}
	\begingroup
	\let\center\flushleft
	\let\endcenter\endflushleft
	\maketitle
	\endgroup
\end{minipage}
\thispagestyle{empty}

\textbf{{\color{red} Insert course description here, as it appears in \textit{Programs and Courses}}}

\begin{tabular}{| p{5cm} | p{10cm} |}
	\hline
	\textbf{Mode of Delivery} & \textbf{{On campus}} \\ \hline
	\textbf{Prerequisites} & \textbf{{\color{red} As listed in \textit{Programs and Courses}}} \\ \hline
	\textbf{Incompatible Courses }& \textbf{{\color{red}As listed in \textit{Programs and Courses}}} \\ \hline
	\textbf{Co-taught Courses} & \textbf{{\color{red}List all courses co-taught with this course	\newline \newline Where co-taught courses include both an undergraduate and a graduate cohort, include the following wording ‘graduate students attend joint classes with undergraduates but are assessed separately’}} \\ \hline
	\textbf{Course Convener:} & \\ \hline
	       & Photo (optional) \\ \hline %\includegraphics{} 
	Phone: &  \\ \hline
	Email: &  \\ \hline
	Office hours for student \newline consultation: &  \\ \hline
	Research Interests & \\ \hline
	Relevant administrator if \newline any (optional) & \\ \hline
	Phone: & \\ \hline
	Email: & [x]@anu.edu \\ \hline
	Lecturer(s) & \\ \hline
	Phone(s): & \\ \hline
	Email(s): & [x]@anu.edu \\ \hline
	Office hours for student \newline consultation: &  \\ \hline
	\textbf{Tutor(s) (optional)} & \textbf{{\color{red}[If the tutors are not known when preparing the guide, a link to a website that will contain this
	information would be appropriate]. If known, add rows for names, and contact details}} \\ \hline
\end{tabular} \\ \\

\centerline{SEMESTER {\color{red}[number or date range of offering if not s1 or s2]}}
\centerline{20XX}

\newpage

\section*{COURSE OVERVIEW}

\subsection*{Course Description (optional)}
\textbf{{\color{red}You may choose to reproduce the course description as published in \textit{Programs and Courses}.}}

\subsection*{Learning Outcomes}
\textbf{{\color{red}Reproduce the course learning outcomes approved by College Education Committee and published in Programs and Courses. The learning outcomes can be numbered, to allow for the linking of outcomes to assessment tasks.}}

\subsection*{Assessment Summary}
\begin{tabular}{| p{2.8cm} | p{2.5cm} | p{2cm} | p{4cm} | p{4cm} |}
	\hline
	\textbf{Assessment Task} & \textbf{Value} & \textbf{Due Date} & \textbf{Date for Return of \newline Assessment} & \textbf{\textit{Linked Learning \newline Outcomes (optional)}} \\ \hline
	1.\textbf{{\color{red}short title of assessment task}} & \textbf{{\color{red}\% or mark, \newline hurdle}} & & & \textbf{\textit{{\color{red}List the number of the relevant outcome(s)}}} \\ \hline
	2. & & & & \\ \hline
	3. & & & & \\ \hline
	\textbf{{\color{red}Insert more rows if needed}} & & & & \\ \hline
	%4. & & & & \\ \hline
	%5. & & & & \\ \hline
\end{tabular}

\subsection*{Research-Led Teaching}
{\bfseries\color{red}Describe in 200 words or fewer the distinctive, research-led features of this course.}

\subsection*{Feedback}
\textbf{Staff Feedback}

{\bfseries\color{red}List the forms of feedback you will give to students in this course (eg written comments, verbal comments, feedback to the whole class, to groups, to individuals, focus groups). You can present this in the form of a bullet point list, with the following sentence stem: 'Students will be given feedback in the following forms in this course:'} \\

\textbf{Student Feedback}

ANU is committed to the demonstration of educational excellence and regularly seeks feedback from students. One of the key formal ways students have to provide feedback is through Student Experience of Learning Support (SELS) surveys. The feedback given in these surveys is anonymous and provides the Colleges, University Education Committee and Academic Board with opportunities to recognise excellent teaching, and opportunities for improvement. \\

For more information on student surveys at ANU and reports on the feedback provided on ANU courses, go to \\
\hphantom{ null } \url{http://unistats.anu.edu.au/surveys/selt/students/} and \\ 
\hphantom{ null } \url{http://unistats.anu.edu.au/surveys/selt/results/learning/}

\subsection*{Policies}
ANU has educational policies, procedures and guidelines, which are designed to ensure that staff and students are aware of the University’s academic standards, and implement them. You can find the University’s education policies and an explanatory glossary at: \url{http://policies.anu.edu.au/} \\

Students are expected to have read the \href{[http://www.anu.edu.au/about/governance/legislation]}{\color{blue} Academic Misconduct Rule} before the commencement of their course. \\

\qquad Other key policies include:

\begin{itemize}
	\item Student Assessment (Coursework)
	\item Student Surveys and Evaluations
\end{itemize}

\subsection*{Required Resources}
{\bfseries\color{red}Commonwealth supported students and domestic full-fee paying students generally must be able to complete the requirements of their program of study without the imposition of fees that are additional to the student contribution amount or tuition fees.} \newline 
{\color{red}Provided that its payment is in accordance with the Act, a fee is of a kind that is into any one or more of the following categories:}

\begin{enumerate}[(a)]
	{\color{red}\item It is a charge for a good or service that is not essential to the course of study.
	\item It is a charge for an alternative form, or alternative forms, of access to a good or service that is an essential component of the course of study but is otherwise made readily available at no additional fee by the higher education provider.
	\item It is a charge for an essential good or service that the student has the choice of acquiring from a supplier other than the higher education provider and is for:
	\begin{enumerate}[(i)]
		\item  equipment or items which become the physical property of the student and are not consumed during the course of study; or
		\item	food, transport and accommodation costs associated with the provision of field trips that form part of the course of study.	
	\end{enumerate}
	\item	It is a fine or a penalty provided it is imposed principally as a disincentive and not in order to raise revenue or cover administrative costs.}
\end{enumerate}

\subsection*{Field trips (optional)}
{\bfseries\color{red}Outline details of field trips required as a component of this course.}

\subsection*{Additional course costs}
{\bfseries\color{red}Include a brief sentence about additional costs that students undertaking this subject incur, such as materials, excursions, travel, accommodation subject to HESA provisions.}

\subsection*{Examination material or equipment}
{\bfseries\color{red}Details about the material or equipment that is permitted in an examination room. If not known at the time of completing the outline, please insert a URL where students will be able to find the relevant information.}

\subsection*{Recommended Resources}

\section*{COURSE SCHEDULE}
{\bfseries\color{red}Be sure to adjust length if teaching period is not the standard semester. Dates need not be inserted in the week column. Where the teaching period is not the standard semester, delete ‘Week’, otherwise delete ‘Session’.}

\begin{tabular}{| p{1.6cm} | p{6cm} | p{6cm} |}
	\hline
	\textbf{Week/ Session} &\textbf{Summary of Activities} & \textbf{Assessment} \\ \hline
	0 & & \\ \hline
	1 & & \\ \hline
	2 & & \\ \hline
	3 & & \\ \hline
	4 & & \\ \hline
	5 & & \\ \hline
	6 & & \\ \hline
	7 & & \\ \hline	
	8 & & \\ \hline
	9 & & \\ \hline
	10 & & \\ \hline
	11 & & \\ \hline
	12 & & \\ \hline
	& Examination Period & \\ \hline
\end{tabular}

\section*{ASSESSMENT REQUIREMENTS}
The ANU is using Turnitin to enhance student citation and referencing techniques, and to assess assignment submissions as a component of the University's approach to managing Academic Integrity. For additional information regarding Turnitin please visit the \href{https://services.anu.edu.au/information-technology/software-systems/turnitin}{\color{blue} ANU Online} website. \\

Students may choose not to submit assessment items through Turnitin. In this instance you will be required to submit, alongside the assessment item itself, copies of all references included in the assessment item. \\

{\color{ForestGreen}
As a further academic integrity control, students may be selected for a 15 minute individual oral examination of their written assessment submissions. \\

Any student identified, either during the current semester or in retrospect, as having used ghost writing services will be investigated under the University’s Academic Misconduct Rule. \\
}

\subsection*{Assessment Tasks}
\textbf{Participation} \\

{\bfseries\color{red}Outline any participation requirements, and (where relevant) criteria by which participation will be judged.} \\

\textbf{Assessment Task 1:} {\bfseries\color{red}insert title} \\

\textbf{Details of task:}

\subsection*{Assessment Rubrics} 

{\bfseries\color{red}If this assessment task is supported by the use of assessment rubrics, please include copies here, or provide a link where students can find them.}

\textbf{Word limit (where applicable):}

\textbf{Value:}

\textbf{Presentation requirements:}

\textbf{Estimated return date:}

\textbf{Hurdle Assessment requirements (where applicable):}

\textbf{Individual Assessment in Group Tasks (where applicable):} \\

\textbf{Assessment Task 2: {\color{red}insert title}} \\

\textbf{Details of task:}

\subsection*{Assessment Rubrics}
{\bfseries\color{red}If this assessment task is supported by the use of assessment rubrics, please include copies here, or provide a link where students can find them.}

\textbf{Word limit (where applicable):}

\textbf{Value:}

\textbf{Presentation requirements:}

\textbf{Estimated return date:}

\textbf{Hurdle Assessment requirements (where applicable):}

\textbf{Individual Assessment in Group Tasks (where applicable):}

\subsection*{Examination(s)}
{\bfseries\color{red}Note if the course includes formal examination and, where necessary, any additional details as applicable to other assessment tasks.}

\subsection*{Assignment submission}
\textbf{Online Submission}: {\bfseries\color{red}Unless an exemption has been approved by the Associate Dean (Education) a submission must be through Turnitin.} Assignments are submitted using Turnitin in the course Wattle site. You will be required to electronically sign a declaration as part of the submission of your assignment. Please keep a copy of the assignment for your records. \\

\textbf{Hard Copy Submission}: {\bfseries\color{red}For some forms of assessment (hand written assignments, art works, laboratory notes, etc.) hard copy submission is appropriate when approved by the Associate Dean (Education). Please state how the students submit such assignments to you via, for example, the physical assignment box. The cover sheet must use the assignment cover sheet template. If your course does not require hard copy submission, delete this sub-section.} Assignments must include the cover sheet available \underline{\color{blue} here}. Please keep a copy of tasks completed for your records.

\subsection*{Extensions and penalties}
\textbf{Extensions and late submission of assessment pieces are covered by the Student Assessment (Coursework) Policy and Procedure.} \\

\textbf{The Course Convener may grant extensions for assessment pieces that are not examinations or take-home examinations. If you need an extension, you must request it in writing on or before the due date. If you have documented and appropriate medical evidence that demonstrates you were not able to request an extension on or before the due date, you may be able to request it after the due date.}\\

{\bfseries\color{red}If you do not accept late submission of assessment, please use the following text:

No submission of assessment tasks without an extension after the due date will be permitted. If an assessment task is not submitted by the due date, a mark of 0 will be awarded. \\

If you accept late submission of assessment, please use the following text: \\

Late submission of assessment tasks without an extension are penalised at the rate of 5\% of the possible marks available per working day or part thereof. Late submission of assessment tasks is not accepted after 10 working days after the due date, or on or after the date specified in the course outline for the return of the assessment item. \\

Late submission is not accepted for take-home examinations.}

\subsection*{Returning assignments}
\textbf{{\color{red}Indicate how student work is to be returned.}}

\subsection*{Resubmission of assignments}
\textbf{{\color{red}State whether students may resubmit some or all assignments and under what conditions.}}

\subsection*{Referencing requirements}
{\bfseries\color{red}State the requirements for referencing, and refer students to the appropriate College or School referencing guide or relevant convention.} \\

{\bfseries\color{red}At this point in the document, you can insert reading lists, class activity specifications and other relevant information.}

\subsection*{Scaling}
Your final mark for the course will be based on the \textbf{raw} marks allocated for each of your assessment items. However, your final mark may not be the same number as produced by that formula, as marks may be \textbf{scaled}. Any scaling applied will preserve the rank order of raw marks (i.e. if your raw mark exceeds that of another student, then your scaled mark will exceed the scaled mark of that student), and may be either up or down.

\subsection*{Privacy Notice}
The ANU has made a number of third party, online, databases available for students to use. Use of each online database is conditional on student end users first agreeing to the database licensor’s terms of service and/or privacy policy. Students should read these carefully. \\

In some cases student end users will be required to register an account with the database licensor and submit personal information, including their: first name; last name; ANU email address; and other information. \\

In cases where student end users are asked to submit ‘content’ to a database, such as an assignment or short answers, the database licensor may only use the student’s ‘content’ in accordance with the terms of service – including any (copyright) licence the student grants to the database licensor. \\

Any personal information or content a student submits may be stored by the licensor, potentially offshore, and will be used to process the database service in accordance with the licensors terms of service and/or privacy policy. \\

If any student chooses not to agree to the database licensor’s terms of service or privacy policy, the student will not be able to access and use the database. In these circumstances students should contact their lecturer to enquire about alternative arrangements that are available.

\subsection*{Tutorial Seminar Registration}
Tutorial signup for this course will be done via the Wattle website. Detailed information about signup times will be provided on Wattle or during your first lecture. When tutorials are available for enrolment, follow these steps:

\begin{enumerate}
	\item  Log on to Wattle, and go to the course site
	\item  Click on the link `Tutorial enrolment'
	\item  On the right of the screen, click on the tab `Become Member of \ldots.' for the tutorial class you wish to enter
	\item  Confirm your choice
\end{enumerate}

If you need to change your enrolment, you will be able to do so by clicking on the tab `Leave group\ldots.' and then re-enrol in another group. You will not be able to enrol in groups that have reached their maximum number. Please note that enrolment in ISIS must be finalised for you to have access to Wattle.

\subsection*{SUPPORT FOR STUDENTS}
The University offers a number of support services for students. Information on these is available online from \url{http://students.anu.edu.au/studentlife/}

\end{document}
